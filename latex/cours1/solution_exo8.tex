\documentclass{article}
\usepackage{fontspec}
\usepackage{xunicode}
\usepackage{polyglossia}
\setmainlanguage{french}
\title{Un titre}%Il manquait un titre
\author{Bibi}%Même si cela n'entraîne pas d'erreur de compilation, la commande \author était en commentaire.

\begin{document}
	\maketitle
%\chapter{Cherchez l'erreur!}Il ne peut pas y avoir de chapitre dans la classe article

\begin{abstract}
La rédactrice de ce document \LaTeX{} a dissimulé sept erreurs; saurez-vous les repérer en vous aidant du fichier .log? 

Corrigez le document jusqu'à obtenir une compilation sans erreur, en indiquant au moyen d'un commentaire ce que vous avez modifié.
\end{abstract}%l'environnement abstract n'était pas fermé

\section*{Introduction}

Lorem ipsum dolor sit amet, consectetuer adipiscing elit. Ut purus elit,
vestibulumut, placerat ac, adipiscing vitae, felis. Curabitur dictumgravida
mauris. Namarcu libero, nonummy eget, consectetuer id, vulputate a, ma-
gna. Donec vehicula augue eu neque. Pellentesque habitant morbi tristique
senectus et netus et malesuada fames ac turpis egestas. Mauris ut leo. Cras
viverra metus rhoncus sem. Nulla et lectus vestibulumurna fringilla ultrices.
Phasellus eu tellus sit amet tortor gravida placerat. Integer sapien est, ia-
culis in, pretiumquis, viverra ac, nunc. Praesent eget semvel leo ultrices
bibendum. Aenean faucibus. Morbi dolor nulla, malesuada eu, pulvinar at,
mollis ac, nulla. Curabitur auctor semper nulla. Donec varius orci eget risus.
Duis nibh mi, congue eu, accumsan eleifend, sagittis quis, diam. Duis eget
orci sit amet orci dignissimrutrum.

\section{Première partie}%Pas de majuscule à la commande \section
Nam dui ligula, fringilla a, euismod sodales, sollicitudin vel, wisi. Morbi
auctor loremnonjusto. Namlacus libero, pretiumat, lobortis vitae, ultricies
et, tellus. Donec aliquet, tortor sed accumsan bibendum, erat ligula aliquet
magna, vitae ornare odio metus a mi. Morbi ac orci et nisl hendrerit mol-
lis. Suspendisse ut massa. Cras nec ante. Pellentesque a nulla. Cumsociis
natoque penatibus et magnis dis parturient montes, nascetur ridiculus mus.
Aliquamtincidunt urna. Nulla ullamcorper vestibulumturpis. Pellentesque
cursus luctus mauris.

\subsection{Une sous-partie}
Nulla malesuada porttitor diam. Donec felis erat, congue non, volut-
pat at, tincidunt tristique, libero. Vivamus viverra fermentumfelis. Donec
nonummy pellentesque ante. Phasellus adipiscing semper elit. Proin fermen-
tummassa ac quam. Sed diamturpis, molestie vitae, placerat a, molestie
nec, leo. Maecenas lacinia. Namipsumligula, eleifendat, accumsannec, sus-
cipit a, ipsum. Morbi blandit ligula feugiat magna. Nunc eleifend consequat
lorem. Sed lacinia nulla vitae enim. Pellentesque tincidunt purus vel magna.
Integer non enim. Praesent euismod nunc eu purus. Donec bibendumquam
in tellus. Nullamcursus pulvinar lectus. Donec et mi. Namvulputate metus
eu enim. Vestibulumpellentesque felis eu massa.

\subsection{Une autre sous-partie} %l'accolade de fin manquait
Quisque ullamcorper placerat ipsum. Cras nibh. Morbi vel justo vitae
lacus tincidunt ultrices. Loremipsumdolor sit amet, consectetuer adipiscing
elit. Inhac habitasse platea dictumst. Integer tempus convallis augue. Etiam
facilisis. Nunc elementumfermentumwisi. Aenean placerat. Ut imperdiet,
enimsedgravida sollicitudin, felis odio placerat quam, ac pulvinar elit purus
eget enim. Nunc vitae tortor. Prointempus nibh sit amet nisl. Vivamus quis
tortor vitae risus porta vehicula.

\end{document}%il manquait cette ligne