\documentclass[12pt]{article}
\usepackage{fontspec}
\usepackage{xunicode}
\usepackage{polyglossia}
\setmainlanguage{french}

\usepackage{reledmac}%Faire une édition critique
\usepackage{reledpar}%Mettre deux textes en vis-à-vis

\begin{document}
	
	
	\section*{Aller plus loin avec reledmac et reledpar}

Pour ces exercices, vous trouverez les textes à utiliser ici:

\href{https://neoclassica.co/2020/05/25/le-poeme-ad-lesbiam-de-catulle-traductions-et-imitations/}{Catulle et ses traductions}

\subsection*{Facile}
Apprenez à vous servir de \verb|reledmac| pour éditer des poèmes (reportez-vous au manuel ou à l'ouvrage de M. Rouquette mis en bibliographie). Numérotez les vers du poème de Catulle  \emph{Ad Lesbiam}.



\subsection*{Intermédiaire}
Mettez en vis-à-vis le poème de Catulle et sa traduction anglaise par Sir Richard Francis Burton.

\subsection*{Difficile}
Créez une commande \verb|\variantes{#1}{#2}| telle que \#1 corresponde au lemme et  \#2 soit le contenu d'une note critique. Testez.
\end{document}