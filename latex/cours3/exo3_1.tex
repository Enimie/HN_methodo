\documentclass[12pt]{book}
\usepackage{fontspec}
\usepackage{xunicode}

%%% attention: problème avec polyglossia dans la texlive 2023
\usepackage{polyglossia}%on peut préciser d'autres langues. La nouvelle version de polyglossia pose probleme avec le style de l'EnC
\setmainlanguage{french}


%\usepackage[citestyle=verbose-ibid]{biblatex}
\usepackage[style=enc,sorting=nyt,maxbibnames=10]{biblatex}%charger le style de l'EnC (téléchargeable ici https://ctan.org/pkg/biblatex-enc)

\addbibresource{exo4.bib}%Mettre le chemin de sa bibliographie (fichier .bib)


\usepackage{csquotes}%nécessaire pour utiliser le style bibliographique, qui utilise enquote



%\defbibheading{}{} Définir les titres des sous-parties de bibliographie

\begin{document}
	%ajouter des références bibliographique: \cite{}, \footcite{}
	%tester les prénotes et postnotes

Lorem ipsum dolor sit amet, consectetuer adipiscing elit. 
Ut purus elit, vestibulumut, placerat ac, adipiscing vitae, felis. 
Curabitur dictumgravida mauris. Namarcu libero,  consectetuer id, vulputate a, magna.


	%imprimer sa bibliographie, ajouter des options (heading, keyword)
\nocite{*}
\printbibliography
\end{document}
