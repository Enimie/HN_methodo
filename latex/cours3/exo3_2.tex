\documentclass[12pt]{book}
\usepackage{fontspec}
\usepackage{xunicode}
\usepackage{polyglossia}
\setmainlanguage{french}

\usepackage{minted}%citer du code


\begin{document}
	Citer du code avec latex. Prérequis:
	\begin{itemize}
		\item avoir python 
		\item avoir avec python la bibliothèque Pygments %https://pygments.org/
	\end{itemize}
	
	Attention, la compilation doit se faire avec l'option -shell-escape passé à la commande xelatex:  configurer  TexStudio
	 
	
	%1/ Décommentez et mettre les codes suivants dans l'environnement minted avec en argument obligatoire le nom du langage
	%3/ linenos et firstnumber=5 pour numéroter et pour commencer la numérotation à la ligne indiquée par firstnumber.
		
\LaTeX: 
	
%commande LaTeX pour indexer  les auteurs mentionnés
%\newcommand{\auteur}[1]{#1\index[npr]{#1}}
	
python:

%#Un commentaire
%def f(x):
% return x**2

	
XML:
	
%<racine>
% <element attribut="valeur de l'attribut">
%  contenu
% </element>
%</racine>
	
	JSON:
	
%var courses = {
% "fruits": [
%   { "kiwis": 3,
%     "mangues": 4,
%     "pommes": null
%   },
%   { "panier": true },
% ]
%};
	
	
\end{document}
